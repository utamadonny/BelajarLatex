\documentclass{article}

\usepackage{amsmath}
\begin{document}
\title{Rangkuman Paper}
\author{Donny Prakarsa Utama}
\maketitle

\cite{Valenti2015} Estimasi orientasi sangat penting untuk mendapatkan feedback yang baik dari controller sensor yang low 
cost pada kendaraan MAV (Micro Aerial Vehicle). 
Tantangannya adalah rendahnya akurasi pada sensor dan 
noisy dari sensor IMU MEMS. Penyelesaian masalah pada paper ini memisahkan problem tilt quaternion dan arah quaternion menjadi 2 sub sistem. Prosedur ini di klaim untuk memisahkan disturbansi magnet roll dan pitch dari medan magnet yang tidak diinginkan. Paper ini merupakan algroritma complementary filter baru untuk 9 DoF

\begin{equation}
^A_B q = [ q_0 ~ q_1~  q_2~  q_3 ]
\end{equation}
\begin{equation}
    ^B_A q^* = ^A_Bq = [ q_0~ -q_1~ -q_2~ -q_3 ]^T
\end{equation}
\begin{equation}
p\otimes q=\left[\begin{matrix}p_0q_0-p_1q_1-p_2q_2-p_3q_3\\p_0q_1+p_1q_0+p_2q_3-p_3q_1\\p_0q_2-p_1q_3+p_2q_0+p_3q_1\\p_0q_3+p_1q_2-p_2q_1+p_3q_0\end{matrix}\right]
\end{equation}
\begin{equation}
    ^Bv_q=^B_A q \otimes ^Av_q \otimes ^B_Aq^*
\end{equation}
\begin{equation}
    v_q = [0 ~ v]^T = [0 ~ v_x ~v_y~v_z]^T
\end{equation}
\begin{equation}
    ^Av_q=^B_Aq^* \otimes ^Bv_q \otimes ^B_Aq = ^A_Bq \otimes ^Bv_q \otimes ^A_Bq^* 
\end{equation}
\begin{equation}
    ^Bv=R(^B_Aq)^Av
\end{equation}
\begin{equation}
    R(^B_Aq)=\left[ 
        \begin{matrix}
            q_0^2+q_1^2-q_2^2-q_3^2 &2(q_1q_2-q_0q_3)&2(q_1q_3-q_0q_2)\\
            2(q_1q_2-q_0q_3)&q_0^2+q_1^2-q_2^2-q_3^2&2(q_2q_3-q_0q_1) \\
            2(q_1q_3+q_0q_2)& 2(q_2q_3+q_0q_1) &q_0^2+q_1^2-q_2^2-q_3^2
        \end{matrix}
    \right]
\end{equation}
\begin{equation}
    ^Av=R(^A_Bq)^Bv=R^T(^B_Aq)^Bv
\end{equation}

Rumus nya sama kayak madgwick filter, karena emang masih sama-sama complementary filter. Yang membedakan perhitungan itu cuma Quaternion from Earth-Field Observations :\emph{ Quaternion from Accelerometer Readings} dan \emph{ Quaternion from Magnetometer Readings}

\cite{Asrofi2015} Paper ini menjelaskan stabilisasi robot berkaki enam (hexapod) dengan sensor IMU 9DoF pada bidang miring berbasis \emph{inverse kinematic}
Menggunakan fuzzy-PID sebagai kendali untuk mempertahankan \emph{body} robot

\bibliography{Reference}
\bibliographystyle{ieeetr}



\end{document}