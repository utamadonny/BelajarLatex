\documentclass[conference]{IEEEtran}
\IEEEoverridecommandlockouts
\usepackage{cite}
\usepackage{amsmath,amssymb,amsfonts}
\usepackage{algorithmic}
\usepackage{graphicx}
\usepackage{url}
\usepackage{textcomp}
\usepackage{xcolor}
\def\BibTeX{{\rm B\kern-.05em{\sc i\kern-.025em b}\kern-.08em
T\kern-.1667em\lower.7ex\hbox{E}\kern-.125emX}}
\begin{document}

\makeatletter
\newcommand{\linebreakand}{%
  \end{@IEEEauthorhalign}
  \hfill\mbox{}\par
  \mbox{}\hfill\begin{@IEEEauthorhalign}
}
\makeatother

\title{xXXXXXxxxXXX
\\ Xsdjaskfjljf}
\author{\IEEEauthorblockN{1\textsuperscript{st} Evelin}
    \IEEEauthorblockA{\textit{XXXXXX Departement} \\
    \textit{Instut XXXXXX}\\
    Jakarta, Indonesia \\
    email@google.com}
    \and
    \IEEEauthorblockN{2\textsuperscript{nd} Sesesese }
    \IEEEauthorblockA{\textit{ Departement} \\
    \textit{IFsfesfnstitut Txexxxxxxxx}\\
    Jakarta, Indonesia \\
    email@google.com}
    \and
    \IEEEauthorblockN{3\textsuperscript{rd} Susi Sustainable}
    \IEEEauthorblockA{\textit{ Departement} \\
    \textit{InstEFitfut exxxgdTekG}\\
    Jakarta, Indonesia \\
    AAAAaaa@xxxxx.ac.id}}
\maketitle


\begin{abstract}
    Penelfafefitian ini afeftermasfeafxxfaek jenis peelitiafffitian awaegadalahpeeggembgangekepribeefgaian bagi siwa kelas X yang beri mater-aeimahaman diri, percaya diri dan tanngjawab. Subjeki cob dalam enelitian pngebanan  , ytu ah materi bimbingan pribadi, ahli media pembelajaran dan guru pembimbing, serta subjek siswa kelas X SMA GAMA Yogyakarta sebagai calon pengguna produk. Penentuan subjek tersebut dilakukan dengan random sampling, yang terdiri dari 10 orang siswa untuk uji coba lapangan utama dan 24 orang siswa untuk uji lapangan operasional. Teknik pengumpulan data dilakukan dengan angket. Angket ini digunakan untuk menilai modul yang dikembangkan dari segi kelengkapan modul, isi materi maupun tampilan fisik modul. Data dianalisis secara deskriptif kuantitatif dan kualitatif.
    Hasil peneliaefaefiange menunjufafgekkan gexahwax,egngaegn utama modul pengsiswa uk fam kGE. Setelagh vegeipengembangan termasuk dalam kategori sangat baik pada uji lapangan operasional. Dengan demikian, modul pengembangan kepribadian bagi siswa kelas X SMA hasil pengembangan ini layak digunakan oleh siswa.
\end{abstract}
\begin{IEEEkeywords}
    yyyyyy, Micrxxxxo-x, Ggggggggid, FFFFFFF.
\end{IEEEkeywords}

\section{Introduction}i
\IEEEPARstart{E}{leaffefctfaefefricafaefitafey} poweggegereggsupgegein - generesGgatgsg  and n planniegegng of elloyed through anticiegegeegatricityf ety supply at certain period of time in an observed area [1]. Load demand depends on different exogenous factorslike temperature, humidity, wind speed, seasonal patternsrelated to human activities and cyclic information [15].
esgagegIn generaeggal, forgagegeacasting methfEFefsgewawgined mulatefasefasf inctors cagusing lgeoarsegical esdatasegeg a of existing area by settto certain grids.[9]. This technique has no issue in the data availability at the onset. However, data availability becomes a problem in developing country and is worsen by the fast demands of electricity supply for improving financial level of the area. Clustering is the process of grouping a set of physical or abstract objects into classes of similar objects. Clustering analysis is a multivariate technique which aims to categorize objects based on their characteristics and solves the issue on large calculation process.

\section{MIxasdsa xLOcsasadAD FORECAST}
\subsection{Serxxxxies MEtode}
Tizzzies can proxxcxcxowsubject change. How tasefsaefaso classifyxxxx accurately has become a hot research point since it is an important element in many computer vision and pattern recognition applications [11]. Micro-spatial load forecast using trending is exploration of historical data and past data growth to predict future load growth. 

Badasdasdxter dasda(dasdasd001) defined timesequence of tdasdasdmeTime  indexed as $FF_{t}$ conceptasddom variable. Se ${F_{1}, 2_{2}, …} or  {…, Z_{0}, Z_{1}, …)$ is generated from stochastic process defined on probabilitydasdmechanisasdasdasdasm, whicdasdasdh is forsadasdasmulated in the dsadasdollowing functasdasdasdasdion:sdasd
\begin{equation}\label{eq:1}
    Z_{t}=F_{t}+e_{t}
\end{equation}

Here, we define Zt as observati234234ontertgergergdt as a set com234234234onent of tfgdsgrsgrend, cyclgdzgzdfvic, seasrgsgrsonality and statistics, and et as error. 
\subsection{Go2eaff34234234maweffpe234235ersdffrtz Met234234af234hod}
Mathemataweagawegafical mefaewfwodel of load efaegforecasting usingaesf Goseafasefmpertz method zgfrgersgtakes growth feature of electricity demands in cer2423235in sm23523alle235235r spa235ce 235te5ds32 to form unexpected ascending line and rea324234234234234 called this a 324234234owing curve ‘S’, in 4234 asymmetric to its points o234234234234234f inflection. Mathematical growth curve according to (Draper and Smith, 1998) is denoted in the following equation: 
\begin{equation} \label{eq:2}
    \omega = \alpha \exp \{-\beta e^{-kt}\}
\end{equation}
with:\\
$\alpha$: Asymptotasdfsdfsdfspproaching infinity
\\$\beta$: Integsadfsadfsdfsadstant
\\$k$: Groasdfsdfsdafe
\\$\omega$: Bosdfsdfasfsdfsdfsdsftimsdfsa  
\\$t$: agsfasffsdfsdfe (dasdfsdfys).
\subsection{Mu234234234ltivarite Si234234234mulation Me234234234242hod}
The divvsdfast of easdasdasdach indeasdpendsdasdasdant pdredicasdastor provides an opportunity to enhrall fordasdasdasdasdecastdsfsd by exploring all viable solutions. Multiple predictors organized in a systematic way called ens34234234emble ne12324235255twork can provide better pr21e12e12e12e12eediction results as co23424pare234234d to a sing352342342le predictor [6].

Spasial Woowafdsdv  alysis is a method of dasd loasadasdd type and density of area asffsevadssadasdlopment based on chsdfsdfsdanges of existing and future land use.
% \begin{figure}fsdf
    
% \end{figure}

\section{CYYYYY For TING}
\subsection{YYYYYY}
Clusteringfy grids into relatively homxxxxxxxxxxxxxxxxxxxxxxxxxxxxxxxxast is implemented in a grid model of its cluster. Cluster similarity s is ecter their similarity.[6] 
\\Euclidean distance matrix is an N x N matrix representing the space between N objects.
\begin{equation}\label{eq:3}
    D=\{d_{ij}\}; ij=1,2,3,4,\dots,N
\end{equation}
Tfsdfsdfhe commsdfsdfsdonly used methosdfsdff faefaefdean distance representing diefce betaefen efaefbjects.
\begin{equation} \label{eq:4}
    =\sqrt{\sum_{k-1}(v_{ik}-v_{jk})^2}
\end{equation}
With 
$d_{oy}$\hspace*{8mm}: Euclidean distance\\
\hspace*{7.5mm}  $G_{xx},x_{xx} $        : Scofsdfsdfre grid -i and ke-j on variable-k

The lowest space xxxxxxxxxxxxxxxxxxxxxxxxxxxxxxxx $(N-1)$ numbers, until all of object stay xxxxxxxxxxxxxresultsxxxxxxustering $C_{j}$. Grouping result and grouping advantage ($C_{j}$ ) are displayed in dendogram. Dendogram can determine toegertetrqeq2etal of clsdfsdfsdfsdfuster andsdfdsfsd its memsdfsdfsdfbers. Clustering profdfdfd function, set in clufefsdfstering process.dsfsdfsdfs The goal is generally to minimize within cluster variance and maximize between clusters variance. In other words, grouping data of similar characteristics into one cluster and grouping data of dissimilar characteristics into other group
\subsection{Principal Component Analysis (PCA)}
PCA is used to preserve important variad in clustdsfsdferfsdfsdfing prsdfsdfsdocess. PCA desdfsdffsdfpends on the type ofsdfsdf origindvdval datxvxvxdaset. If orighave the same units, the principle component is derived from c12e12eopo corr2e1e12e12eation ati2eq2e12e1oata into2e12e12estan12e1d for2e1e12e12em:  
\begin{equation} \label{eq:5}
   Zii_{isadj} = \frac{x_{woo}-\bar{x_{jhh}}}{s_{jiw}}
\end{equation}
With $\bar{x_{1}}$:  variable masdasdeans -j \\
\hspace*{7.5mm}  $s_{2}$:  variableasdasd stanasdasdard devidsadqwe1233ation -j

Total of principxxxxxxxxxxxxxxxxxxxxxxxxxxxxxxxxxxxxx cumulative variance of 75\% or more than total variance.


\subsection{Factor Analysis}
Factor anathe structure of xxxxxxxxfactors in lxxoxwxrxxxnxxuxxmber thaxn the obxsez1x23czcved vawo analysecommon factor analysis. Th corres and principle component variables is written as a function as follows:
\begin{equation}\label{eq:6}
    r_{ij}=\alpha_{ij}\sqrt{\lambda_{j}}
\end{equation}
Where   $r_{ij}$ : Corxxxelxxatixxon coeffixxxcient axxxmongxx variables
\\ \hspace*{1cm}    $\lambda_{j}$ :   Total vdfsdfsdfsdfned

\subsection{Mathematical Model}
The model is fordfdfdfsdfsdfsdfsdfsthe following mathemscsatical model [2]
\begin{equation}
    Y=b_{1}+b_{2}X_{2}+b_{3}X_{3}+\dots+b_{k}X_{k}+e
\end{equation}
Simpldsfsdfsdfsdfsdfo matrix: 
\begin{equation}\label{eq:8}
    Y=Xb+e 
\end{equation}
\begin{equation*}
Y=
\begin{bmatrix}
  Y_{1000}\\H_{2}\\Z_{3}\\.\\.\\.\\Y_{n}  
\end{bmatrix}
b=
\begin{bmatrix}
    x_{1}\\s_{2}\\g_{3}\\.\\.\\.\\b_{k} 
\end{bmatrix}
e=
\begin{bmatrix}
    x_{1}\\y_{2}\\z_{3}\\.\\.\\.\\e_{n} 
\end{bmatrix}
\end{equation*}


\begin{equation}\label{eq:9}
\sum{e_{i}^2} =e'e=(Y-Xb)'(Y-Xb)
\end{equation} 

\subsection{xxxxx Variables}
To obtain load densasdasdasdasevious load density, we needasdasdasdasariable (except landdsadadasdasdasfsdsdfgsgs absowerwrlute pewerwerrcentage edsfsdfsdfsdsdgsr (MAPE).[10]

Trend of land use change xxxx from spatxxxxal plannxxxxegional plannxxxfsxx With envadfsfironmensal the data of neisfsdfsdfsdfmlet were range between thesdfsdfsdfsdf-10 years, we refer to trend of regional development in past years as to determine land use change.
\subsection{Pefsdfsadfk Losfsfsddfsdfad Fosfsdfsdfsdfecast}
Asdfsdfftfer tsdfsdrfefsdnd of dsfsfsdefach varisdfsdffsdable isdf obtasdfinedsexcept fodfsdr land fssdf usdfsdsing RT/RW and historical data), varsdiblsdffefsds trendfsdfsd gsdowtfh are used fsdfsdfsdfsdfsdfor forecasting the load density fdsfsdfsdfsdollowing past load density model.
sdfsdf
Rsdfultsdfsdff ldfoaddfs defsnsity sdfsdrecast fsdfat efsdachsdyeasdfr ardfe obtainedsdfsdfsfromsdf tf cfsdlusfsdter. sdFurtsdfer, fsdfthe rfsdults ardsfsde fsdused to calculate load density of respective sector of similar cluster. With the density of each sector, we can estimate the capacity of power by sector an obtained by multipying load density per sector ector of its ddstrictastheasdusasd. Meanwdhilesdas,  changeasa of length er proculr of district by summing 
    P_{Totalofneighbdfsgsfsdgergaforhood}(t)=Cwefewf_{f}(P_{Rwefwef}(t)+qefweP_{wewB}(twerwe)+P_werweI(werwert)+P_S(t))
\end{equation}
where  $C_{f}$ : Coincwefwefweident  fafewghyjuytkktor 
\subsection{Flowchaerherthrthrt of microsrhttrehtrtthpatial lorhtrdhrad fogkukurufjyjrecast }
rherthre is ferthretherlowchart of all methhrthehods, disperherthrlayed arehtrhrs follerhtetherows:
% \begin{figure}
    
% \end{figure}
\section{RExxxSULT AxxND DISCxxxUSSION}
\subsection{xxxxx}
Clustgdfgfdhttjytyu8ering procgrsegeyss invoasdasfdslvesik a wide rangefdsf oagergargf electricidgvariables adfgergnd ndfgofv-edfgectrivdfbvdfbdcity variabledgsdfgdfgs. Total of variables are 12, and totasdf
    
% \end{table}

\subsection{Buixxxxxxxldxing Cluxxxxxster }

e argertakegegeneras ergthe objeergergct of researrgceh is geggerarea netwergegeork including some parts of Tangerang, Bogor and West Jakarta. There are 114 districts taken as grids for clustering. According to data mining, two methods of clustering are hierarchical clustering and non-hierarchical clustering [6]. Hierarchical clustering is algorithm that group two or more similar objects into cluster.  The process is repeatedly executed until clusters merged together producing tree-like diagram, which shows the hierarchical relationship between objects. Dendogram is usually the output to describe the hierarchical process. [8].
erga
Clurggrewgedfserirng igs aempaloyegaefad by groupfinefg objeawfcfats (diswefawefwects) ifnto cluswterfwasfin whichawfweevery cluster consists of district with relatively homogenous characteristics. Objects grouping are implemented by clustering technique. 

Agglomewfwefweferative clusterifwefwefwefng starts from N cwfefwefluster to singleton cluster, in which N is the total ofe data, wafhwefawle eaweffdivisive clustering starts from singleton cluster into N cluster. The process can be seen in the figure below: 
% \begin{figure}
    
% \end{figure}
Frowefwefwefwefm reswefwefwefwefweffewefults of clxusxxtxxering,x the totwexfwefwefx1efwwefwef1xefwefdisxxctsfewefwwef is grouped intewfwefwefwefo 5 clusters, as seen below:
% \begin{table}
    
% \end{table}
\subsection{Cwefwefweflwefuswefwefwefter Characteristics}
When clustwefWEFRWGHTRHRJUKYUering is applied, every cluster hYU relaJYJtively homxYJenxous ccRHteristHDHics forHTD each district, as depicted below:
% \begin{table}
    
% \end{table}
Hierarchical clustering is ablsdfsafa into one hiedsafafesiccccxxher grxxxtics’ illustratioxxxxxxxter at each dimenxxxxxxxxxxsion, we use discriminant analysis. Procedure started after cluster is determined.

\subsection{Calfeculation Of asdch Disasdasdtrict}
Baseefad onwgtaxawxx eafch xxvxiaxxle fdsf(excgt for lgerand ugergerge xWexd to forecast load density. Further process is to calculate total power of districts by summing x enexxxxxxrgy per secxcxxxxxxy and socxxxxxxial). 
% \begin{table}
    
% \end{table}
% \begin{table}
    
% \end{table}
Results of energy grSDFSDFSDFSDowth arwqereww elaborated Ffrom lSDFoad per sector, which demands in industxxxxxxxxxxst energy demands.
We can see that avwerefqwfage percentage ofgrwgerdustrial growth igwerwqroreerge dynamic compared to other sector loads. System growth increases withxxxxxxxxge of xqq21139 \%. Average growxxxxxtors; housing 7.05 \%, industry 6.63 \%, business 6.0 \%,xxxxxxxxxxxx 6 \%.

\section{Conclusion}
MicDASDASDro-spatiaASASD
DuADSDASASDe to its easy proAASDDASDASDgrASDammingASDsystem, this ASDASque is feasible toDA perfected by eASDASDASnhancing itsDASDASD ability in patter recognition through fuzzy application or other intelligent systems.
DAD
\section*{Reference}
\noindent
[1]	MASDASDASujiati DwAS KDASDASDrtikasari and AriASDASDASf RohmadASASDASD ASDPrayogio(2018) “Demand Forecasting of Electricity in Indonesia with Limited Historical”, J. Phys.: Conf Data.

\noindent
[2]	X.asdsa Sasdasdasun, Zasdasd. OuyASDASDsdgdfgASDsdada D. YWDQWDASue (23424234017) "Short-ASDASDTerm Load ForeASDASDASasting BasedSADASD oASDASDASDn MultiASDASDASariate Linear Regression," IEEE Conference on Energy Internet and Energy System Integration (EI2), Beijing.

\noindent
[3] JP Carvadasdasaasddllo, PH asdasdasdfsgddarsen, AH Sanstad, CA Goldman ( 2017), "Load Forecasting in Electric Utility Integrated Resource Planning," osti.gov 6. 

\noindent
[4] Laurffqeqnec, qfwfewfqteffqr & Lucka, Masdasdwaria.(2017) “New clustering-based forecasting method for disaggregated end-consumer electricity load using smart grid data”. 210-215. 10.1109/INFORMATICS.8327248.

\noindent
[5] Krzygszfafeatof Gasfdfsaajownadfaficzek, Toxxxxxxxkowski (2018) “ Simulawefwefwefergergftion Study on hes For m Electricity Forecasting”, Complexity, Complex Optimization and Simulation in Power System.

\noindent
[6] Raza, Muhammrajah, Midasthulanming & Lee, Kwang, "Multisdarasdiatasde Enseeasdcasor Demand Prediction of Anomalous Days”. IEEE Transactions on Sustainable Energy, p. 6, 2018

\noindentdas
[7] Jianwei Mi, Lyinfsadasdg Qiu (2018) “Shorxxxxxdad  asdBadasasdon”, Hindasdas, MathematicdasdalEngineering.
Improved Exponasdaentiadal Smoo2017.

\noindent
[8] Shahzadeh, Abbas & Khosrai, Saeid (2015) “Impxxxxxs using smart meter data”. 1-7. 10.1109/IJCNN. 7280393.

\noindent
[9] Asadasdasd Sendsadsadasden (2013), “Study of Msadasdicrossabfghergre4dasdpatial Electecasdasdast Bassadasdasded on Land Use Sasdasdimulation” Journal of Electrical Media.]

\noindent
[10] Asadasdasddri Sedasdasdnen, Titi fas-Term utoasdasdasfsregrdasdessive Integradsadasdasted Moving Avethod ”, SUTET Sic Journal, Vol. 7, No.2.

\noindent
[3]

\noindent
[3]

\noindent
[3]

\noindent
[3]

\end{document}