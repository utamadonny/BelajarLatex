\documentclass[10pt]{IEEEtran}
\usepackage{biblatex}

\bibliography{References} 

\begin{document}
\title{Resume : Perbedaan Metode dan Metodologi Penelitan}
\author{Donny Prakarsa Utama\\3332170032\\ Matakuliah Seminar}
\maketitle

\section{Metodo logi Penelitian}
Metodologi Penelitian adalah rangkaian progress yang sistematis untuk menyelesaikan masalah\cite{youtube:1}.
\section{Metode Penelitian}
Metode Penelitian adalah cara atau proses yang dipilih untuk menyelesaikan suatu masal1ah dalam sebuah penelitian. Metode penelitian merupakan bagian dari langkah sistematis dalam menyelesaikan masalah (metodologi penelitian)\cite{youtube:1}.
\subsection{Rancangan Penelitian} Rancangan penelitian adalah sketsa atau kerangka yang didesain oleh peneliti sebagai
\textit{(research plan)}  . Rancangan penelitian dibuat untuk mendapatkan persetujuan untuk melakukan penelitian disebut proposal penelitian\cite{youtube:1}.\\
Cakupan:

1. Jenis Penelitian. \\ Contohnya penelitian kualitatif atau kuantitatif.

2. Desain Penelitian (jika ada) \\
Metode apa yang dipakai. Desain dalam penelitian yg dibuat.

3. Prosedur Penelitian (jika ada) \\
Penjabaran dari desain penelitian yang diajukan. Dibuat flowchartnya.


\subsubsection*{Contoh}
Jenis penelitian ini merupakan jenis penelitian deskriptif kuantitatif, karena
dalam penelitian ini mendeskripsikan keadaan yang terjadi pada saat sekarang
secara sistematis dan faktual dengan tujuan untuk memaparkan serta
penyelesaian dari masalah yang diteliti \cite{unila:2}.


\subsection{Populasi dan Sampel}
Populasi dan sampel dalam metode penelitian berbicara mengenai bagaimana peneliti mendapatkan partisipan atau data \cite{youtube:1}. \\
Cakupan:

1. Identifikasi dan batasan populasi.f

2. Prosedur dan teknik pengambilan sampel.

3. Besarnya sampel

\subsubsection*{Contoh}
Populasi dari
penelitian ini adalah seluruh pengurus dan anggota Komite Sekolah di
SD Fransiskus yang berjumlah 13 orang. Informan yang diperlukan
untuk mempertajam data yang diperoleh dalam hal ini adalah Kepala
Sekolah SD Fransiskus Pringsewu.
Dalam penelitian ini menggunakan
penelitian populasi karena subjeknya kurang dari 100 yaitu berjumlah 14
orang. Sesuai dengan pendapat di atas, maka seluruh jumlah populasi
dijadikan sebagai sampel penelitian\cite{unila:2}.


\subsection{Instrumen Penelitian}
Pada bagian ini peneliti menjabarkan instrumen/alat yang digunakan untuk mendapakan data. Misalnya alat ukur, sensor, angket dsb\cite{youtube:1}.
\\
Syarat instrumen:

1. Validitas

2. Reliabelitas
\subsubsection*{Catatan} Harus melakukan uji validitas dan reliabelitas instrumen yang kita buat. Kecuali, menggunakan data dari penelitian orang yang validitas dan reliabilitasnya telah diketahui\cite{youtube:1}.

\subsubsection*{Contoh}
Instrumen yang dipergunakan disesuaikan dengan tahapan yang dilakukan
dalam penelitian sebagai berikut\cite{unila:2}.
Pada tahap pengembangannya instrumen yang dipakai antara lain:

1. Angket disusun atas dasar konsultasi ke ahli materi komite sekolah dan
pembimbing.

2. Observasi disusun atas dasar data sekunder dan konsultasi dengan
pembimbing.

3. Dokumentasi disusun atas dasar data sekunder dan konsultasi dengan
pembimbing

\subsection{Pengumpulan Data}Pengumpulan data dilakukan untuk dapat menjawab rumusan masalah. Keterhubungan metode pengumpulan data bergantung pada rumusan masalah yang dibuat \cite{youtube:1}.
\subsubsection*{Contoh}
Pengumpulan data yang dilakukan dalam penelitian ini melalui beberapa
teknik dengan maksud untuk mendapatkan data yang lengkap guna
menunjang permasalahan yang nantinya dapat mendukung keberhasilan
dalam penelitian ini. Teknik yang digunakan yaitu angket, observasi dan
dokumentasi\cite{unila:2}.

\subsection{Analisis Data}
Analisis Data menjelaskan tentang teknik analisis data yang digunakan dalam penelitian untuk menghasilkan data penunjang penelitian. Metode analisis data dapat dilakukan secara manual atau komputer(Excel, SPSS,  Matlab dsb). Metode analisis data bergantung pada metode penelitian yang digunakan \cite{youtube:1}.

\subsubsection*{Contoh}Berdasarkan data yang telah diperoleh melalui angket makan teknik
pengelolaan data atau analisis data yang dipergunakan adalah data kuantitatif
dan kualitatif. Hasil pengolahan data, disajikan dalam bentuk tabel untuk
mempresentasekan hasil analisis dengan menggunakan teknik deskriptif
kualitatif\cite{unila:2}.

\newpage

\printbibliography

\end{document}
