\documentclass{article}
\usepackage{indentfirst}
\begin{document}
\title{Course Note}
\maketitle
\section{Intro to Personal Data}
\subsection{What is Cybersecurity?}
The connected electronic information network has become an integral part of our daily lives. All types of organizations, such as medical, financial, and education institutions, use this network to operate effectively. They utilize the network by collecting, processing, storing, and sharing vast amounts of digital information. As more digital information is gathered and shared, the protection of this information is becoming even more vital to our national security and economic stability.

Cybersecurity is the ongoing effort to protect these networked systems and all of the data from unauthorized use or harm. On a personal level, you need to safeguard your identity, your data, and your computing devices. At the corporate level, it is everyone’s responsibility to protect the organization’s reputation, data, and customers. At the state level, national security, and the safety and well-being of the citizens are at stake.

\subsection{Your Online and Offline Identity}

As more time is spent online, your identity, both online and offline, can affect your life. Your offline identity is the person who your friends and family interact with on a daily basis at home, at school, or work. They know your personal information, such as your name, age, or where you live. Your online identity is who you are in cyberspace. Your online identity is how you present yourself to others online. This online identity should only reveal a limited amount of information about you.

You should take care when choosing a username or alias for your online identity. The username should not include any personal information. It should be something appropriate and respectful. This username should not lead strangers to think you are an easy target for cybercrimes or unwanted attention.

\subsection{Your Data}

Any information about you can be considered to be your data. This personal information can uniquely identify you as an individual. This data includes the pictures and messages that you exchange with your family and friends online. Other information, such as name, social security number, date and place of birth, or mother‘s maiden name, is known by you and used to identify you. Information such as medical, educational, financial, and employment information, can also be used to identify you online.

\subsection*{Medical Records}

Every time you go to the doctor’s office, more information is added to your electronic health records (EHRs). The prescription from your family doctor becomes part of your EHR. Your EHR includes your physical health, mental health, and other personal information that may not be medically-related. For example, if you had counseling as a child when there were major changes in the family, this will be somewhere in your medical records. Besides your medical history and personal information, the EHR may also include information about your family.

Medical devices, such as fitness bands, use the cloud platform to enable wireless transfer, storage and display of clinical data like heart rates, blood pressures and blood sugars. These devices can generate an enormous amount of clinical data that could become part of your medical records.

\subsection*{Education Records}

As you progress through your education, information about your grades and test scores, your attendance, courses taken, awards and degrees rewarded, and any disciplinary reports may be in your education record. This record may also include contact information, health and immunization records, and special education records including individualized education programs (IEPs).
\subsection*{Employment and Financial Records}

Your financial record may include information about your income and expenditures. Tax records could include paycheck stubs, credit card statements, your credit rating and other banking information. Your employment information can include your past employment and your performance.

\subsection{Where is Your Data?
}

All of this information is about you. There are different laws that protect your privacy and data in your country. But do you know where your data is?

When you are at the doctor’s office, the conversation you have with the doctor is recorded in your medical chart. For billing purposes, this information may be shared with the insurance company to ensure appropriate billing and quality. Now, a part of your medical record for the visit is also at the insurance company.

The store loyalty cards maybe a convenient way to save money for your purchases. However, the store is compiling a profile of your purchases and using that information for its own use. The profile shows a buyer purchases a certain brand and flavor of toothpaste regularly. The store uses this information to target the buyer with special offers from the marketing partner. By using the loyalty card, the store and the marketing partner have a profile for the purchasing behavior of a customer.

When you share your pictures online with your friends, do you know who may have a copy of the pictures? Copies of the pictures are on your own devices. Your friends may have copies of those pictures downloaded onto their devices. If the pictures are shared publicly, strangers may have copies of them, too. They could download those pictures or take screenshots of those pictures. Because the pictures were posted online, they are also saved on servers located in different parts of the world. Now the pictures are no longer only found on your computing devices.

\subsection{Your Computing Devices}

Your computing devices do not just store your data. Now these devices have become the portal to your data and generate information about you.

Unless you have chosen to receive paper statements for all of your accounts, you use your computing devices to access the data. If you want a digital copy of the most recent credit card statement, you use your computing devices to access the website of the credit card issuer. If you want to pay your credit card bill online, you access the website of your bank to transfer the funds using your computing devices. Besides allowing you to access your information, the computing devices can also generate information about you.

With all this information about you available online, your personal data has become profitable to hackers.
\section{Personal Data as Target}
\subsection{They Want Your Money}



If you have anything of value, the criminals want it.

Your online credentials are valuable. These credentials give the thieves access to your accounts. You may think the frequent flyer miles you have earned are not valuable to cybercriminals. Think again. After approximately 10,000 American Airlines and United accounts were hacked, cybercriminals booked free flights and upgrades using these stolen credentials. Even though the frequent flyer miles were returned to the customers by the airlines, this demonstrates the value of login credentials. A criminal could also take advantage of your relationships. They could access your online accounts and your reputation to trick you into wiring money to your friends or family. The criminal can send messages stating that your family or friends need you to wire them money so they can get home from abroad after losing their wallets.

The criminals are very imaginative when they are trying to trick you into giving them money. They do not just steal your money; they could also steal your identity and ruin your life.
\subsection{They Want Your Identity}


Besides stealing your money for a short-term monetary gain, the criminals want long-term profits by stealing your identity.

As medical costs rise, medical identity theft is also on the rise. The identity thieves can steal your medical insurance and use your medical benefits for themselves, and these medical procedures are now in your medical records.

The annual tax filing procedures may vary from country to country; however, cybercriminals see this time as an opportunity. For example, the people of the United States need to file their taxes by April 15 of each year. The Internal Revenue Service (IRS) does not check the tax return against the information from the employer until July. An identity thief can file a fake tax return and collect the refund. The legitimate filers will notice when their returns are rejected by IRS. With the stolen identity, they can also open credit card accounts and run up debts in your name. This will cause damage to your credit rating and make it more difficult for you to obtain loans.

Personal credentials can also lead to corporate data and government data access.

\section{Organizational Data}
\subsection{Types of Organizational Data}
\subsection*{Traditional Data}

Corporate data includes personnel information, intellectual properties, and financial data. The personnel information includes application materials, payroll, offer letters, employee agreements, and any information used in making employment decisions. Intellectual property, such as patents, trademarks and new product plans, allows a business to gain economic advantage over its competitors. This intellectual property can be considered a trade secret; losing this information can be disastrous for the future of the company. The financial data, such as income statements, balance sheets, and cash flow statements of a company gives insight into the health of the company.
\subsection*{Internet of Things and Big Data}

With the emergence of the Internet of Things (IoT), there is a lot more data to manage and secure. IoT is a large network of physical objects, such as sensors and equipment that extend beyond the traditional computer network. All these connections, plus the fact that we have expanded storage capacity and storage services through the cloud and virtualization, lead to the exponential growth of data. This data has created a new area of interest in technology and business called “Big Data". With the velocity, volume, and variety of data generated by the IoT and the daily operations of business, the confidentiality, integrity and availability of this data is vital to the survival of the organization.

\subsection{Confidentiality, Integrity, and Availability}


Confidentiality, integrity and availability, known as the CIA triad (Figure 1), is a guideline for information security for an organization. Confidentiality ensures the privacy of data by restricting access through authentication encryption. Integrity assures that the information is accurate and trustworthy. Availability ensures that the information is accessible to authorized people.
\subsection*{Confidentiality}


Another term for confidentiality would be privacy. Company policies should restrict access to the information to authorized personnel and ensure that only those authorized individuals view this data. The data may be compartmentalized according to the security or sensitivity level of the information. For example, a Java program developer should not have to access to the personal information of all employees. Furthermore, employees should receive training to understand the best practices in safeguarding sensitive information to protect themselves and the company from attacks. Methods to ensure confidentiality include data encryption, username ID and password, two factor authentication, and minimizing exposure of sensitive information.

\subsection*{Integrity}

Integrity is accuracy, consistency, and trustworthiness of the data during its entire life cycle. Data must be unaltered during transit and not changed by unauthorized entities. File permissions and user access control can prevent unauthorized access. Version control can be used to prevent accidental changes by authorized users. Backups must be available to restore any corrupted data, and checksum hashing can be used to verify integrity of the data during transfer.

A checksum is used to verify the integrity of files, or strings of characters, after they have been transferred from one device to another across your local network or the Internet. Checksums are calculated with hash functions. Some of the common checksums are MD5, SHA-1, SHA-256, and SHA-512. A hash function uses a mathematical algorithm to transform the data into fixed-length value that represents the data, as shown in Figure 2. The hashed value is simply there for comparison. From the hashed value, the original data cannot be retrieved directly. For example, if you forgot your password, your password cannot be recovered from the hashed value. The password must be reset.

After a file is downloaded, you can verify its integrity by verifying the hash values from the source with the one you generated using any hash calculator. By comparing the hash values, you can ensure that the file has not been tampered with or corrupted during the transfer.

\subsection*{Availability}

Maintaining equipment, performing hardware repairs, keeping operating systems and software up to date, and creating backups ensure the availability of the network and data to the authorized users. Plans should be in place to recover quickly from natural or man-made disasters. Security equipment or software, such as firewalls, guard against downtime due to attacks such as denial of service (DoS). Denial of service occurs when an attacker attempts to overwhelm resources so the services are not available to the users.

\subsection{The Consequences of a Security Breach}


To protect an organization from every possible cyberattack is not feasible, for a few reasons. The expertise necessary to set up and maintain the secure network can be expensive. Attackers will always continue to find new ways to target networks. Eventually, an advanced and targeted cyberattack will succeed. The priority will then be how quickly your security team can respond to the attack to minimize the loss of data, downtime, and revenue.

By now you know that anything posted online can live online forever, even if you were able to erase all the copies in your possession. If your servers were hacked, the confidential personnel information could be made public. A hacker (or hacking group) may vandalize the company website by posting untrue information and ruin the company’s reputation that took years to build. The hackers can also take down the company website causing the company to lose revenue. If the website is down for longer periods of time, the company may appear unreliable and possibly lose credibility. If the company website or network has been breached, this could lead to leaked confidential documents, revealed trade secrets, and stolen intellectual property. The loss of all this information may impede company growth and expansion.

The monetary cost of a breach is much higher than just replacing any lost or stolen devices, investing in existing security and strengthening the building’s physical security. The company may be responsible for contacting all the affected customers about the breach and may have to be prepared for litigation. With all this turmoil, employees may choose to leave the company. The company may need to focus less on growing and more on repairing its reputation


\section{Security Breach Example 1}

The online password manager, LastPass, detected unusual activity on its network in July 2015. It turned out that hackers had stolen user email addresses, password reminders, and authentication hashes. Fortunately for the users, the hackers were unable to obtain anyone’s encrypted password vaults.

Even though there was a security breach, LastPass could still safeguard the users’ account information. LastPass requires email verification or multi-factor authentication whenever there is a new login from an unknown device or IP address. The hackers would also need the master password to access the account.

LastPass users also have some responsibility in safeguarding their own accounts. The users should always use complex master passwords and change the master passwords periodically. The users should always beware of Phishing attacks. An example of a Phishing attack would be if an attacker sent fake emails claiming to be from LastPass. The emails ask the users to click an embedded link and change the password. The link in the email goes to a fraudulent version of the website used to steal the master password. The users should never click the embedded links in an email. The users should also be careful with their password reminder. The password reminder should not give away your passwords. Most importantly, the users should enable multi-factor authentication when available for any website that offers it.

If the users and service providers both utilize the proper tools and procedures to safeguard the users’ information, the users’ data could still be protected, even in the event of security breach.


Security Breach Example 2

The high tech toy maker for children, Vtech, suffered a security breach to its database in November 2015. This breach could affect millions of customers around the world, including children. The data breach exposed sensitive information including customer names, email addresses, passwords, pictures, and chat logs.

A toy tablet had become a new target for hackers. The customers had shared photos and used the chat features through the toy tablets. The information was not secured properly, and the company website did not support secure SSL communication. Even though the breach did not expose any credit card information and personal identification data, the company was suspended on the stock exchange because the concern over the hack was so great.

Vtech did not safeguard the customers’ information properly and it was exposed during the breach. Even though the company informed its customers that their passwords had been hashed, it was still possible for the hackers to decipher them. The passwords in the database were scrambled using MD5 hash function, but the security questions and answers were stored in plaintext. Unfortunately, MD5 hash function has known vulnerabilities. The hackers can determine the original passwords by comparing millions of pre-calculated hash values.

With the information exposed in this data breach, cybercriminals could use it to create email accounts, apply for credits, and commit crimes before the children were old enough to go to school. For the parents of these children, the cybercriminals could take over the online accounts because many people reuse their passwords on different websites and accounts.

The security breach not only impacted the privacy of the customers, it ruined the company’s reputation, as indicated by the company when its presence on the stock exchange was suspended.

For parents, it is a wake-up call to be more vigilant about their children’s privacy online and demand better security for children’s products. For the manufacturers of network-connected products, they need to be more aggressive in the protection of customer data and privacy now and in the future, as the cyberattack landscape evolves.


\end{document}
