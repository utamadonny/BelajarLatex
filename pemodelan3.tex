\documentclass[10pt]{article}
\usepackage{tikz}
\usepackage[pdf]{graphviz}
\usepackage{amsmath}

\begin{document}
\title{Tugas 3 : Soal Pemodelan dan Simulasi }
\author{Donny Prakarsa Utama\\3332170032\\Pemodelan dan Indentifikasi Sistem-1}
\maketitle

\section{Latihan 1}
\begin{enumerate}
    \item Anda tentu pernah melihat dalam toko busana (pakaian), patung (boneka) dengan pakaian yang dikenakan.
    \begin{enumerate}
        \item Coba sebutkan jenis model (fisik atau abstrak) patung tersebut.
        
        Apa yang digantikan (diwakili) oleh patung tersebut?        \begin{cases}
            x+2y=3\\
            2x+y=4
        \end{cases}del baju, termasuk model fisik atau abstrakkah model tersebut?
        \item Sebelum suatu busana dibuat, biasanya dibuat dahulu draf busana (gambar pola busana). Termasuk jenis model fisik atau abstrakkah draf busana tersebut?
    \end{enumerate}
    \item Anda tentunya pernah melihat peragawati atau peragawan yang sedang berlenggak-lenggok memperagakan busana.
\begin{enumerate}
    \item Termasuk model fisik atau abstrakkah peragawati atau peragawan tersebut?
    
    Siapa yang diwakili oleh peragawati atau peragawan tersebut?
\item Proses apa yang diperagakan oleh peragawati atau peragawan
tersebut?
\end{enumerate}
\item Permainan perang-perangan yang dilakukan dua kelompok orang.
\begin{enumerate}
    \item Orang-orang dan senjata tiruan yang digunakan dapat digolongkan sebagai model fisik atau abstrak?
\item Permainan dan strategi yang dilakukan oleh setiap kelompok apakah dapat disebut dengan model perang?
\item Proses apa yang dilakukan dalam permainan tersebut?
\end{enumerate}
\item Permainan perang-perangan yang dilakukan dengan menggunakan komputer (tentu saja dengan menggunakan perangkat lunak), melalui layar komputer.
\begin{enumerate}
    \item Termasuk model fisik atau abstrakkah pihak-pihak yang terlibat dalam perang-perangan ini?
    \item  Proses apa yang dilakukan oleh orang yang menjalankan komputer dengan perangkat lunak perang-perangan tersebut?
\end{enumerate}
\end{enumerate}

\section*{Jawaban Latihan 1}
\begin{enumerate}
    \item 
    \begin{enumerate}
        \item Model fisik. Patung tersebut menggantikan (mewakili) orang.Patung tersebut berperan memperagakan busana. Apabila
menggunakan orang maka biayanya lebih mahal.
\item Apabila busana yang dikenakannya belum dijual (baru sebagai
contoh, belum diproduksi dalam jumlah banyak), maka busana yang
dipakaikan pada patung dapat disebut dengan model busana. Model
ini disebut dengan model abstrak.
\item Model abstrak.
    \end{enumerate}
    \item 
\begin{enumerate}
    \item Model fisik.
Peragawati atau peragawan mewakili calon pengguna (pembeli
yang akan memakai) busana.
\item Proses menjalankan peniruan perlakuan (simulasi) terhadap model.
\end{enumerate}
\item  \begin{enumerate}
    \item  Model fisik.
\item Ya.
\item Proses menjalankan peniruan perlakuan (simulasi) terhadap model.
\end{enumerate}
\item \begin{enumerate}
    \item Gambar orang serta gambar senjata yang digunakan dalam
komputer merupakan model abstrak
\item Proses menjalankan peniruan (simulasi) perang-perangan. Oleh
karena proses yang dilakukan berbantuan komputer, maka proses
ini disebut juga dengan simulasi berbantuan komputer (computer
simulation)
\end{enumerate}
\end{enumerate}

\section{Tes Formatif 1 dan Jawaban}
\begin{enumerate}
    \item Model berikut ini dapat digolongkan sebagai jenis model fisik:
    \begin{enumerate}
        \item Baju contoh (sebelum diproduksi dalam jumlah banyak \textbf{\emph{Jawab: Benar}} 
        \item Orang yang melakukan reka ulang kejadian di TKP (Tempat
        Kejadian Perkara)\textbf{\emph{Jawab: Benar}}
        \item Peta TKP (Tempat Kejadian Perkara)\textbf{\emph{Jawab: Salah}}
        \item Rumus Pitagoras\textbf{\emph{Jawab: Salah}}
        \item Foto bola dunia\textbf{\emph{Jawab: Salah}}
        \item Gambar rumah tinggal\textbf{\emph{Jawab: Salah}}
        \item Maket kampus\textbf{\emph{Jawab: Benar}}
        \item Persamaan kesetimbangan\textbf{\emph{Jawab: Salah}}
        \item Robot yang dikirim ke luar angkasa\textbf{\emph{Jawab: Benar}}
        \item Miniatur menara Eifel\textbf{\emph{Jawab: Benar}}
    \end{enumerate} 
\item Manakah yang sedang melakukan pemodelan?
Seseorang yang sedang: 
\begin{enumerate}
    \item memainkan suatu permainan menggunakan komputer\textbf{\emph{Jawab: Salah}}
    \item memikirkan bagaimana menurunkan rumus akar persamaan kuadrat\textbf{\emph{Jawab: Benar}}
\item menjahit busana\textbf{\emph{Jawab: Salah}}
\item membuat peta kelurahan tertentu\textbf{\emph{Jawab: Benar}}
\item menyusun persamaan reaksi kimia\textbf{\emph{Jawab: Benar}}
\item menggambar denah rumah\textbf{\emph{Jawab: Benar}}
\item menyusun anggaran biaya tahunan\textbf{\emph{Jawab: Benar}}
\item melakukan perjalanan keliling dunia\textbf{\emph{Jawab: Salah}}
\item membeli barang berdasarkan rencana anggaran pembelian\textbf{\emph{Jawab: Salah}}
\item pembuatan miniatur jembatan\textbf{\emph{Jawab: Benar}}

\end{enumerate}\end{enumerate}
\section{Latihan 2}
\begin{enumerate}
    \item Model abstrak berikut ini manakah yang dapat dianggap sebagai model
matematis: \begin{enumerate}
    \item $\frac{dx}{dt} = 2x dan \frac{dy}{dx}= x+y$ 
    \item $2H_2+O_2 \rightarrow 2H_2O$
    \item $y+x^2 – 4x= 0$
    \item $CH_2 + 2O_2 \rightarrow CO_2 + 2H_2O$
    \item $ 2a+b=4 dan b+a=5$
    \item $S\leftarrow B ; B \leftarrow A ; A \leftarrow S$
    \item $\frac{dx}{dt}=k(x-10)$
\end{enumerate}
\item Sebutkan jenis model matematis berikut ini:

\begin{enumerate}
    \item $y=x^2-4x$
\item  $x+y=10 dan x – y=0$
\item  $y'' – 2y'+y=0$
\item  $\frac{dy}{dx}=k(y+5)$
\item $\frac{dx}{dt} 2x+y dan \frac{dy}{dt}=x – y$


\end{enumerate}Untuk nomor 3, 4, dan 5, nyatakan model matematis berikut penjelasan
awalnya.
\item Uang kuliah tahun ini naik satu juta rupiah dibanding tahun sebelumnya. Uang kuliah tahun sebelumnya adalah satu setengah juta rupiah.
\item Uang kuliah tahun ini naik satu juta rupiah dibanding tahun sebelumnya.
\item Untuk membeli semangkok mie bakso dan dua gelas es campur
diperlukan uang sebanyak tiga puluh ribu rupiah. Sedangkan untuk
membeli dua mangkok mie bakso dan segelas es campur diperlukan
uang sebanyak empat puluh ribu rupiah. 
\end{enumerate}
\section*{Jawaban Latihan 2}
    \begin{enumerate}
        \item a,c,e, dan g
        \item \begin{enumerate}
            \item persamaan kuadrat, fungsi kuadrat
            \item sistem persamaan linear
            \item persamaan diferensial (biasa)
            \item persamaan diferensial
            \item sistem persamaan diferensial
        \end{enumerate}
        \item x: uang kuliah tahun ini (dalam jutaan rupiah)
        maka model matematisnya: x=1,5+1
        \item y: uang kuliah tahun ini (dalam jutaan rupiah)
        
        x: uang kuliah tahun sebelumnya,
        maka model matematisnya: y x 1
        \item x: harga semangkok mie bakso (dalam puluhan ribu rupiah)
        
        y: harga segelas es campur (dalam puluhan ribu rupiah)
        maka model matematisnya:
        \begin{cases}
            x+2y=3\\
            2x+y=4
        \end{cases}
    \end{enumerate}

    \section{Tes Formatif 2}
    \begin{enumerate}
        \item Periksalah pernyataan di bawah ini:
\begin{enumerate}
    \item $2y – 3x=0$ adalah persamaan linear
    \item $\frac{dy}{dx}=y$ adalah  persamaan diferensial
    \item $x^2-y=2x-5$ adalah fungsi persamaan kuadrat
\end{enumerate}
\item Periksalah pernyataan di bawah ini:
\begin{enumerate}
    \item $U(t)=3x-5$ adalah fungsi linear
    \item $f(x,y)=x^2+3x^2y+y^2-xy+5$ adalah fungsi kuadrat
    \item $P(t)=10e^-x$ adalah fungsi exponensial
\end{enumerate}
\item Dinyatakan bahwa x: banyaknya uang Ali (dalam jutaan rupiah).
\begin{enumerate}
    \item Jika Ali belanja barang sebanyak sejuta rupiah, maka uang Ali menjadi $x-1$
    \item Jika Ali diberi uang sebanyak sejuta rupiah, maka uang Ali menjadi $x+1$
    \item Jika Ali kehilangan uang sebanyak sejuta rupiah, maka uang Ali
    menjadi $x-1$
\end{enumerate}
\item Dinyatakan bahwa x: umur Ali (dalam tahun), y: umur Badu (dalam
tahun).\begin{enumerate}
    \item Apabila Ali dua tahun lebih tua dari Badu, maka $x=y–2$
    \item Apabila Ali lebih tua dari Badu, maka $x>y$
    \item Apabila umur Ali lahir dua tahun lebih dahulu dari Badu, maka $x=y+2$
\end{enumerate}
\item Dari rumah ke kampus harus melalui dahulu terminal bus. Jarak dari
rumah ke terminal bus adalah 2 km lebih dekat daripada jarak dari
terminal bus ke kampus.\begin{enumerate}
    \item Apabila dinyatakan bahwa x: jarak dari rumah ke terminal bus
    (dalam km), y: jarak dari terminal bus ke kampus (dalam km),
    maka $x=y–2$
    \item Apabila dinyatakan bahwa p: jarak dari terminal bus ke kampus
    (dalam km), q: jarak dari rumah ke terminal bus (dalam km), maka
    $q=p+2$
    \item Apabila dinyatakan bahwa $x_1$ : jarak dari rumah ke kampus (dalam
    km), $x_2$ : jarak dari rumah ke terminal bus (dalam km), maka
    $x_1-2x_2 =2$
\end{enumerate}
\item Jarak dari rumah ke kampus adalah a km. Apabila x: kecepatan rata-rata
Ali bersepeda (dalam per jam). Apabila Ali ingin agar waktu tempuhnya
adalah \begin{enumerate}
    \item 30 menit, maka $x=2a$
    \item 45 menit, maka $x=\frac{3}{4}a$
    \item 20 menit, maka $x=3a$
\end{enumerate}
\item Harga 1 liter bensin adalah Rp. 4500, apabila x: banyaknya uang untuk
membeli bensin (dalam rupiah), y: banyaknya bensin yang akan dibeli
(dalam liter), maka ....
\begin{enumerate}
    \item $y=\frac{4500}{x}$
    \item $x=4500y$
    \item $\frac{x}{y}=4500$
\end{enumerate}
\item Apabila x: banyaknya uang Ali (dalam jutaan rupiah),
y: banyaknya uang Badu (dalam jutaan rupiah),
\begin{enumerate}
    \item $x+y+5$, menyatakan jumlah uang Ali dan Badu adalah lima juta
    rupiah
    \item $x – y=1 $, menyatakan selisih uang Ali dan Badu adalah satu juta
    rupiah
    \item $x+y=5 atau x–y= 1$ adalah sistem persamaan linear untuk
    menentukan banyaknya uang Ali dan uang Badu.
\end{enumerate}
\item Pernyataan $y=x+1$ merupakan\dots
\begin{enumerate}
    \item hubungan banyaknya uang Ali dan Badu, jika x: banyaknya uang Ali
    (dalam jutaan rupiah), y: banyaknya uang Badu (dalam jutaan
    rupiah).
    \item hubungan antara 2 bilangan, jika x: bilangan pertama, y: bilangan
    kedua
    \item hubungan umur Ali dan Badu, jika x: umur Ali (dalam tahun) dan y:
    umur Badu (dalam tahun)
\end{enumerate}
\item Diberikan pernyataan $y=ax$
\begin{enumerate}
    \item Apabila a: harga satuan barang (dalam rupiah), x: banyak barang
    yang dibeli, maka y menyatakan uang (dalam rupiah) yang harus
    dibayarkan
    \item Apabila a: kecepatan bersepeda (km/jam), x: jarak yang ditempuh
    (km), maka y menyatakan waktu tempuh
    \item Apabila a: tetapan pertumbuhan populasi penduduk, x: waktu
(dalam tahun), maka y menyatakan besarnya populasi penduduk
(dalam ribuan) dari tahun ke tahun
\end{enumerate}
    \end{enumerate}
    \section*{Jawaban Tes Formatif 2}
    Note:\\
    Untuk soal nomor 1 sampai dengan 10, berikanlah jawaban
    \\A. jika pernyataan 1 dan 2 benar
    \\B. jika pernyataan 1 dan 3 benar
    \\C. jika pernyataan 2 dan 3 benar
    \\D. jika pernyataan 1, 2, dan 3 benar
    \begin{enumerate}
        \item D
        \item B
        \item D
        \item A
        \item D
        \item B
        \item C
        \item A
        \item D
        \item B
    \end{enumerate}
\end{document}
