\documentclass[10pt]{article}
\usepackage{tikz}
\usepackage[pdf,png]{graphviz}
\usepackage[png]{graphicx}
\usepackage{amsmath}
\usepackage{physics}

\begin{document}
\title{Tugas 5 : Soal Pemodelan dan Simulasi }
\author{Donny Prakarsa Utama\\3332170032\\Pemodelan dan Indentifikasi Sistem-1}
\maketitle

Dari 2 artikel berikut,
\begin{itemize}
\item Pengidentifikasian Parameter Fungsi Alih Sistem Pada Plant Simulasi Orde Tiga Dan Empat Dengan Metode Algoritma Genetik
\item Sistem Identifikasi Model Dinamik Proton Exchange Membrane Fuel Cell (Pemfc) Dalam Struktur Blok Wiener
\end{itemize}
\begin{enumerate}
\item cari permasalahan
\item metode dihubungkan paramterik dan nonparamterik
\item Hasil
\end{enumerate}

Jawab:
\begin{enumerate}
\item Permasalahan:

Studi kasus pertama ialah prediksi laju kendaraan menggunakan filter moving average dan ANN.
Diketahui,
F(t) = gaya yang dihasilkan oleh mobil = 80t N yang diasumsikan.
V = kecepatan ($m/s$)
M = massa kendaraan = $1000 kg$
b = koefisien gesekan = $40 N_s/m$
\begin{figure}
    \centering
    \includegraphics[width=40mm]{modelmobil}
    \caption{Gambar Model}
\end{figure}
\item dengan metode non paramterik ANN
Dengan Hukum Newton Aksi-Reaksi
$$F-F_gesek=m\alpha$$
$$F-bv=m\alpha$$
$$F=m \dv{v}{t}+bv$$
jadi,
$$80t=1000 \dv{v}{t}+40v$$
sehingga laplace-nya
$$ V(S)=\frac{2}{s^2}-\frac{50}{s}+\frac{50}{s+0.04}$$
inverse laplace
$$ v(t)=2t-50+50e^{-0.04t} $$

Dengan Matlab :
\begin{verbatim}
clear;
% Define and plot time vs velocity for the model
t=[0:.5:30]';
v=2*t-50+50*exp(-0.04*t);
grid
figure(1);
plot(t,v);
xlabel("time in seconds");ylabel("speed in m/s");
 
%Define and plot training as well as checking data
data=[t v];
trndata=data(1:2:size(t),:);
chkdata=data(2:2:size(t),:);
grid
figure(2);
plot(trndata(:,1),trndata(:,2),"o",chkdata(:,1),chkdata(:,2),"x");
xlabel("time in seconds");ylabel("speed in m/s");
%initialize ANN
trainpoint=trndata(:,1)';
trainoutput=trndata(:,2)';
net=newff(minmax(trainpoint),[10 1],{'tansig' 'purelin'});
%Simulate and plot the network ouput without training
Y=sim(net,trainpoint);
grid
figure(3);
plot(trainpoint,trainoutput,trainpoint,Y,"o");
%Initialize parameter
net.trainParam.epochs=100;
net.trainParam.goal=0.0001;
%Train the network and plot the output
net=train(net,trainpoint,trainoutput);
Y=sim(net,trainpoint);
figure(4);
plot(trainpoint,trainoutput,trainpoint,Y,"o");
%test the network and plot the error
checkpoint=chkdata(:,1)';
checkoutput=chkdata(:,2)';
W=sim(net,checkpoint);
e=W-checkoutput;
grid
figure(5);
plot(e);
\end{verbatim}

\item Hasil
\begin{figure}[!htb]
    \centering
    \includegraphics[width=200]{nkontinyu}
    \caption{Grafik Waktu Kontinyu}
\end{figure}

\begin{figure}[!htb]
    \centering
    \includegraphics[width=200]{barbelajarANN}
    \caption{Bar Pembelajaran ANN}
\end{figure}

\begin{figure}[!htb]
    \centering
    \includegraphics[width= 300]{modelANN}
    \caption{Model ANN}
\end{figure}

\end{enumerate}


\end{document}
